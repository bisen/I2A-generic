\documentclass[10pt, twocolumn]{article}
\usepackage{geometry}[margin=1in]
\usepackage{graphicx}

\title{Deep Reinforcement Learning for Atari Games}
\author{Garnet Liu \and Vikram Saraph \and Birol Senturk}

\begin{document}
\maketitle

\begin{abstract}
In recent years, deep learning techniques have been successfully adapted and applied to problems
in reinforcement learning. In this project, we survey and evaluate one such technique, called \emph{imagination augmented agents},
(or \emph{I2A}). This technique was initially developed and published by Weber \emph{et. al.} at DeepMind \cite{}. In their original
work, the authors describe the novel I2A as one that combines aspects of \emph{model-free} learning, in which an agent makes decisions
based only on its current observations, with \emph{model-based} learning, where the agent attempts to learn about its environment
in addition to a policy. In their work, the model-free agent is augmented with a model-based \emph{imagination engine}, which
makes predictions about future trajectories of the agent from its current state.

While model-free agents have seemingly more flexible architectures, in practice they do not generalize do different tasks. Incorporating model-based aspects allows one to provide additional fine-tuning to the network's architecture in some way that is representative of the agent's environment. Indeed, the authors of the original paper demonstrate a significant improvement in the agent when adding imagination to a standard model-free agent. They evaluate I2A on two different 2d games: MiniPacman, which is a simplified version of PacMan, and Sokoban, in which the agent must push blocks onto designated targets. 



\end{abstract}

\section{Introduction}

\section{Related Work}

\section{Background}

\section{Dataset}
We use the Arcade Learning Environment (ALE) to simulate playing Atari games. ALE has been integrated
into OpenAI Gym, which provides a standardized interface for taking actions and retrieving the resulting new
state, reward, etc. The game's state is taken to be the RGB image of the game at a given point in time. See
a screenshot below.

\section{Methodology}

\section{Results}

\section{Discussion}

\end{document}